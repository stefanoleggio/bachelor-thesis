\chapter{Attacco all'autenticazione delle reti 2G-4G}
Le reti cellulari dal 2G al 4G condividono lo stesso schema architetturale, per questo gran parte delle vulnerabilità che vengono
utilizzate negli attacchi di tipo \textit{denial of service} sono comuni.
Ci sono numerosi modi per effettuare un attacco \gls{dos} all'autenticazione, alcuni già accennati nella sezione 4.1.4, in questo capitolo verranno messe in pratica 
nelle reti 2G fino al 4G.\\
Fondamentalmente, in modo da creare un \textit{denial of service} nel \textit{core network} di una rete cellulare tramite una richiesta di identificazione bisogna forzare
la computazione dei vettori di autenticazione in modo tale da fare sprecare risorse computazionali all'infrastruttura cellulare.
Nel momento che un dispositivo si collega alla rete cellulare si possono verificare le seguenti casistiche:
\begin{itemize}
    \item Se il dispositivo ha una \gls{sim} valida inizio la procedura di autenticazione.
    \item Se il dispositivo non ha una \gls{sim} valida inizio la procedura di autenticazione ma senza consumare abbastanza risorse del \textit{network}.
    \item Se il dispositivo non ha una \gls{sim} la procedura di autenticazione non viene iniziata.
\end{itemize}
Quindi, è chiaro che per effettuare un \gls{dos} al sistema di autenticazione degli utenti è necessario disporre o simulare dei dispositivi con delle \gls{sim} valide. La validità della \gls{sim} è 
in primo luogo controllata dalla presenza di un \gls{imsi} valido come spiegato nella sezione precedente.
Di seguito verranno trattate le principali metodologie per effettuare un \gls{dos} al sistema di autenticazione.

\clearpage

\section{Botnet}
Il metodo più conosciuto per creare un \textit{denial of service} a una rete cellulare è tramite una \textit{botnet}.
In questo modo, l'attaccante ha a disposizione un elevato numero di dispositivi con \gls{sim} valida che hanno la possibilità di effettuare massivamente una procedura
di autenticazione causando delle dispendiose computazioni all'interno del \textit{network}.\\
In \cite{measuring-dos} è descritto come effettuare un \gls{ddos} a una rete cellulare di tipo 2G/3G in modo da esasperare di richieste il suo componente più critico: l'\gls{hlr}.
Con 11750 dispositivi infettati è possibile degradare le performance della \gls{hlr} del 93\%\cite{measuring-dos}, garantendo quindi un quasi totale malfunzionamento dell'infrastruttura.\\
Questa tipologia di attacco è molto pericolosa, e spesso anche la più comune, non è pero esente da diverse problematiche: prima di tutto risulta facilmente rilevabile da un sistema di 
monitoraggio della rete. Inolte, è richiesto un numero molto elevato di dispositivi, sopratutto se si tiene presente che questi devono 
appartenere alla stessa zona di competenza della \gls{hlr}.\\

\section{IMSI \textit{catching}}
Un metodo alternativo all'utilizzo di una \textit{botnet} è avere a disposizione un \textit{database} di \gls{imsi} rubati per effettuare un \textit{flooding} di richieste di autenticazione.\\
Dato che nelle reti 2G-4G l'\gls{imsi} viene trasmesso in chiaro al momento dell'autenticazione, riuscire a ottenerli è abbastanza semplice.
In \cite{imsi-catcher} vengono citati i modi più comuni per appropriarsene per poi utilizzarli in un 
attacco.
\begin{figure}[h]
    \centering
    \includegraphics[width=0.5\textwidth]{images/imsi-catcher.jpg}
    \caption{Strumento per rubare IMSI}
\end{figure}\\
Per rubare l'\gls{imsi} si mette in pratica un attacco di tipo \textit{Man In The Middle} (MITM), intromettendosi nella comunicazione fra il \gls{ms} e la \gls{bs}.\\
Nelle reti di seconda generazione questo risulta molto semplice poichè, come spiegato nella sezione 5.1, l'\gls{imsi} viene trasmesso in chiaro se il \gls{ms} è la prima volta che si connette al registro di quella specifica
zona. Inoltre, dato che nel \gls{gsm} l'autenticazione non è mutua è possibile creare una \textit{fake basestation} e collezzionare tutti gli \gls{imsi} dei dispositivi che si connettono.
Sono stati introdotti diversi identificativi temporanei come il \gls{tmsi} per fare in modo che l'\gls{imsi} non debba essere inviato in ogni procedura di autenticazione, ma sono tutti facilmente raggirabili poichè cambiano con una
frequenza troppo bassa.\\
In \cite{dos-imsi} viene illustrato un metodo per ottenere gli \gls{imsi} di qualsiasi dispositivo nello standard UMTS nonostante l'autenticazione mutua. Infatti viene spiegato come basti mandare al \gls{ms} una \textit{user identity request} impersonandosi 
la \gls{vlr} e il \gls{ms} risponderà con il proprio \gls{imsi} in chiaro.
\begin{figure}[h]
    \centering
    \includegraphics[width=0.5\textwidth]{images/imsi-catch-umts.png}
    \caption{IMSI \textit{catching} nelle reti UMTS\cite{dos-imsi}}
\end{figure}\\

\clearpage

\section{Attacco alle reti con dispositivi SIM-less}
In \cite{umts-dos} e \cite{gsm-dos-simless} sono descritti degli attacchi all'autenticazione degli utenti utilizzando dispositivi senza una \gls{sim} commerciale, ma bensì delle interfacce di comunicazione 
programmabili. Questo è stato fatto perchè utilizzare dei \gls{ms} come dispositivi per effettuare un attacco \gls{dos} rappresenta un fattore limitante in termini di prestazioni. Infatti, 
i sistemi operativi dei \gls{ms} impongono degli intervalli di tempo fra una richiesta e un'altra.\\
Entrambi gli attacchi dimostrano che è possibile causare un \gls{dos} con un numero di dispositivi senza \gls{sim} molto minore rispetto allo stato dell'arte. 
\subsection{GSM}
É stato necessario analizzare la rispettiva \textit{air interface} del \gls{gsm} per valutare quale è il numero massimo di richieste di autenticazione 
che possono essere inviate al secondo a una \textit{base station}. Questa misurazione risulta di fondamentale importanza poichè riesce anche a fornire il numero necessario 
di dispositivi per raggiungere il massimo delle \gls{tps}.
Nell'immagine seguente vengono illustrati i messaggi e i canali in cui viaggiano durante l'autenticazione alla rete \gls{gsm}.
\begin{figure}[h]
    \centering
    \includegraphics[width=0.5\textwidth]{images/gsm-air-channel.png}
    \caption{Messaggi scambiati durante l'autenticazione in una rete GSM\cite{gsm-dos-simless}}
\end{figure}

\clearpage

\subsection{UMTS}
L'immagine seguente rappresenta un semplice schema del dispositivo con \gls{sim} programmabile per effettuare un \gls{dos} a una rete \gls{umts}\cite{umts-dos}.
\begin{figure}[h]
    \centering
    \includegraphics[width=0.5\textwidth]{images/umts-dos-device.png}
    \caption{Dispositivo per l'attacco DOS alle reti UMTS\cite{umts-dos}}
\end{figure}\\
Come è stato fatto per la rete \gls{gsm}, è stato necessario analizzare l'\textit{air interface} dell'\gls{umts} per valutare il numero di \gls{tps}.
Nell'immagine seguente vengono illustrati i messaggi e i canali in cui viaggiano durante l'autenticazione alla rete \gls{umts}.
\begin{figure}[h]
    \centering
    \includegraphics[width=0.5\textwidth]{images/umts-air-channel.png}
    \caption{Messaggi scambiati durante l'autenticazione in una rete UMTS\cite{umts-dos}}
\end{figure}\\
Nelle reti \gls{umts} è stato calcolato che il limite più stringente di TPS durante la comunicazione con la \textit{base station} è dato dal canale \gls{fach} con 28 \gls{tps}.
Questo, ha portato a concludere che bastano 446 dispositivi per effettuare una notevole degradazione del sistema, molti di meno rispetto degli 11K necessari per una \textit{botnet}\cite{dos-imsi}.\\
Nello stesso articolo è spiegato come è possibile duplicare le prestazioni dell'attacco usando delle \gls{sim} valide. In questo modo infatti, i vettori di autenticazione vengono generati una seconda volta se si segnala al
\textit{network} che l'\gls{autn} calcolato non risulta corretto.
