\chapter{Conclusioni}
In questo documento sono state analizzate le più comuni vulnerabilità che consentono di effettuare un 
attacco di tipo \textit{denial of service} alle reti cellulari. In particolare, sono state analizzate le vulnerabilità nelle autenticazioni 
di tutte le generazioni.\\
Dopo un'attenta analisi dei meccanismi di autenticazione e delle classiche vulnerabilità che vengono usate nelle generazioni 2G-4G, si può finalmente
fare un confronto fra la sicurezza dell'ultima generazione 5G e quelle precedenti.\\
Il 5G ha sicuramente apportato dei consistenti miglioramenti di sicurezza, come ampiamente trattato riguardo la cifratura dell'\gls{imsi}.\\
Nonostante ciò, è innegabile che in questa ultima generazione gli attacchi \gls{dos} saranno molto più semplici da realizzare, ma sopratutto più pericolosi dati
i compiti sensibili che alcuni dispositivi connessi a questa rete dovranno svolgere.