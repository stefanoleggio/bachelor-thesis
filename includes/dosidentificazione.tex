\section{Attacco Denial of Service al sistema di identificazione}
\subsection{Denial of Service}
L'attacco di tipo \textit{Denial of Service} (DOS) consiste nel rendere non disponibili servizi offerti da computer o altri
dispositivi \cite{dos-definition}. Questo avviene esasperando di richieste la macchina o infrastruttura che viene scelta come
vittima. La risorse della vittima verranno quindi interrogate in modo massivo fino al punto di indurre il sistema al collasso.\\
Una variante dell'attacco DOS è il \textit{Distributed Denial of Service} (DDOS), in cui l'attaccante non è composto solamente da una sola
macchina, ma bensì da una rete intera chiamata \textit{botnet}. Questa seconda versione è più difficile da realizzare ma al tempo stesso
molto più efficacie. Solitamente, la \textit{botnet} è composta dagli \textit{zombies}, ovvero dispositivi di utenti normali ignari del fatto 
di essere stati infettati da un \textit{malware} che consente all'attaccante di averne il controllo.\\
%Immagine DDos
Le reti cellulari non sono esenti da questo tipo di attacchi, anzi, sono una delle tipologie più frequenti e sopratutto difficile da risolvere
poichè le vulnerabilità che sfruttano sono organiche nell'architettura della rete.

\subsection{Identificazione}
Il meccanismo di identificazione è la procedura per verificare che un determinato dispositivo
è abilitato a connettersi alla rete. Il 5G apporta una modifica organica del suo funzionamento, per
questo si è deciso di analizzare nelle successive sezioni solamente i meccanismi di autenticazione della rete
UMTS e 5G.\\
L'identificazione rappresenta un meccanismo molto vulnerabile ad attacchi di tipo \textit{Denial of Service} poichè, 
in alcuni casi, si riesce a consumare delle onerose operazioni computazionali anche con dispositivi che non sono abilitati,
quindi senza SIM.
\subsubsection{UMTS}
Il meccanismo presente nella rete UMTS è lo stesso usato nelle generazioni cellulari GSM, GPRS e EDGE. Inoltre, il 
suo funzionamento è molto simile a quello delle reti LTE (4G), pertanto si è deciso di presentarlo solamente una volta.\\
Un \acrshort{ms} che si vuole collegare alla rete deve procedere con la fase di autenticazione o identificazione anche detta \textit{Authentication and key agreement}
(AKA). In questa fase, viene interrogata la rispettiva HLR/AuC dove l'IMSI del dispositivo viene validato, se tutto procede correttamente
viene notificato il SGSN che inoltra al \acrshort{ms} l'avviso di autenticazione completata.
%Schema autenticazione umts
\subsubsection{5G}
\subsection{Risultati}