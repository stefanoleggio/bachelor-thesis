\chapter{Generazioni cellulari}
Nel corso degli anni, si sono susseguite diverse generazioni di tecnologie cellulari che hanno apportato
notevoli cambiamenti alla loro architettura e infrastruttura per consentire il raggiungimento di prestazioni migliori\cite{architecture-evolution}.\\
Di seguito verranno presentate le principali caratteristiche
delle diverse generazioni cellulari, in modo tale da rendere di facile comprensione l'analisi dei meccanismi
di autenticazione che verranno approfonditi nelle prossime sezioni.\\
Oltre ad elencare le principali caratteristiche di ogni generazione verranno analizzate nel dettaglio le specifiche  
delle architetture.
\begin{figure}[ht]
    \centering
    \includegraphics[width=0.7\textwidth]{images/generations-scheme.jpg}
    \caption{Schema delle generazioni cellulari}
\end{figure}

\clearpage

\section{1G}
La generazione 1G è uno dei primi standard di comunicazione cellulare. Il suo funzionamento era completamente analogico 
e ormai è stata rimpiazzata totalmente dalle generazioni digitali successive.\\
L'architettura di questa generazione è molto semplice, è composta da tre componenti principali:
\begin{itemize}
    \item Antenne per la trasmissione
    \item \gls{mtso}
    \item Unità mobile (cellulare)
\end{itemize}
\begin{figure}[ht]
    \centering
    \includegraphics[width=0.7\textwidth]{images/1g.jpg}
    \caption{Architettura 1G}
\end{figure}
Si basava sulla \gls{fdma} in cui ogni dispositivo che si connetteva alla stazione radio
aveva assegnata una specifica sotto banda\cite{generations}.

\clearpage

\section{2G}
A differenza della prima generazione, la seconda introuduce per la prima volta una rete completamente digitale.
Questa tecnologia cellulare è composta da diverse versioni che si sono susseguite nel corso degli anni aggiungendo nuove 
funzionalità.
Anche la sua architettura subisce delle modifiche, per questo verranno trattate separatamente in seguito.
\subsubsection{GSM}
Il \gls{gsm}\cite{gsm} è uno standard di seconda generazione che introuduce importanti novità.\\
Le principali caratteristiche introdotte sono:
\begin{itemize}
    \item Maggiori velocità di trasmissione
    \item Cifratura della comunicazione
    \item Introduzione di nuovi servizi come gli \gls{sms}
\end{itemize}
\begin{figure}[ht]
    \centering
    \includegraphics[width=0.7\textwidth]{images/2g-gsm.jpg}
    \caption{Architettura GSM}
\end{figure}
La sua architettura è composta da due macro aree: La \gls{bss} e la \gls{nss}.
Il \gls{bss} è l'insieme delle antenne ricevitori che rappresentano il primo collegamento con il \gls{ms}, mentre il \gls{nss} rappresenta il \textit{core network} del \gls{gsm}.\\
Il \gls{nss} è formato dai seguenti componenti:
\begin{itemize}
    \item \gls{msc} è l'elemento centrale dell'atchitettura \gls{gsm}, si occupa di interfacciare le \gls{bs} con la rete telefonica \gls{ptsn}.
    \item \gls{hlr} \textit{database} centrale che contiene informazioni inerenti a tutti i \textit{subscribers}, molte delle informazioni
    che contiene sono dei puntatori agli archivi seguenti.
    \item \gls{vlr} \textit{database} che memorizza la posizione degli utenti.
    \item \gls{eir} \textit{database} degli \gls{imei} dei dispositivi. Grazie a questo archivio è possibile creare delle \textit{blacklist}
    per evitare l'accesso a determinati terminali.
    \item \gls{auc} \textit{database} delle informazioni di sicurezza associate agli utenti registrati.
\end{itemize}

\clearpage

\subsection{GPRS}
La rete \gls{gprs}\cite{gprs-edge} introduce per la prima volta un trasferimento dati a commutazione di pacchetto per rendere 
possibile l'utilizzo dei servizi \textit{internet} con il proprio dispositivo cellulare\cite{gsm-architecture}.
La sua architettura è la stessa di quella del \gls{gsm} ma con dei componenti aggiuntivi che consentono la trasmissione dei pacchetti. 
Per esempio, il \gls{sgsn} è un componente per la gestione dei dispositivi connessi alla rete.
\begin{figure}[ht]
    \centering
    \includegraphics[width=0.7\textwidth]{images/2g-gprs.png}
    \caption{Architettura GPRS}
\end{figure}
\subsection{EDGE}
Evoluzione del \gls{gprs} che consente maggiori velocità, l'architettura resta invariata\cite{gprs-edge}.

\clearpage

\section{3G}
L'architettura della terza generazione riprende quella già vista nella seconda. Infatti, questa generazione ha avuto come principale obbiettivo 
quello di consolidare l'integrazione della rete internet nei sistemi cellulari ed aumentare la velocità di trasmissione per consentire l'utilizzo 
di nuovi servizi.\\
L'accesso al canale radio avviene con la tecnologia \gls{wcdma} con canale di banda 5 MHz.

\subsection{UMTS}
Lo \gls{umts} è il primo standard di terza generazione.
La sua architettura è composta dai seguenti elementi principali:
\begin{itemize}
    \item \gls{msc}, componente che ha la stessa funzione di quello in 2G. Questa volta il \gls{vlr} è integrato al suo interno.
    \item \gls{hlr}/\gls{auc} e \gls{eir}
    \item \gls{sgsn} e \gls{ggsn} ovvero dei componenti ripresi dalla rete \gls{gprs} per la commutazione a pacchetto.
\end{itemize}
\begin{figure}[ht]
    \centering
    \includegraphics[width=0.7\textwidth]{images/3g-umts.png}
    \caption{Architettura UMTS}
\end{figure}


\subsection{HSPA/HSPA+}
Evoluzione del \gls{umts} per consentire velocità maggiori apportando modifiche nella trasmissione del segnale.
Con questo nuovo standard si riescono a raggiungere velocità di 42 Mb/s\cite{hspa}.

\clearpage

\section{4G}
La quarta generazione è al momento quella più utilizzata, permette di avere dei servizi basati su velocità molto alte. 
A differenza delle precedenti generazioni che dovevano gestire due \textit{core network}: uno per la rete telefonica e un altro
per \textit{internet}, per la prima volta il 4G introduce un unico \textit{core network} basato su \gls{ip}.\\
Per consentire un aumento consistente della velocità, le maggiori modifiche di questa generazione sono state apportate nella \textit{radio interface}, mentre
l'architettura rimane con una struttura simile a quella precedente.
\subsection{LTE}
la \gls{lte} è uno standard di quarta generazione che ha i seguenti componenti architetturali\cite{lte}:
\begin{itemize}
    \item \gls{hss} è il \textit{database} centrale dei \textit{subscriber} come l'\gls{hlr} del \gls{gsm}/\gls{umts}.
    \item \gls{mme} è il corrispettivo del \gls{vlr} in \gls{gsm}/\gls{umts}.
    \item \gls{sgw} è un componente che svolge il ruolo di \textit{router} indirizzando i dati dalla \textit{base station}
    al P-GW.
    \item \gls{pgw} è il componente per interfacciare il \textit{core network} con \textit{internet}.
    \item \gls{pcrf} è un componente responsabile delle regole di gestione per il flusso di informazioni.
\end{itemize}
\begin{figure}[ht]
    \centering
    \includegraphics[width=0.8\textwidth]{images/4g-lte.jpg}
    \caption{Architettura LTE}
\end{figure}

\clearpage

\section{5G}
Il 5G, ovvero lo standard di quinta generazione rappresenta l'ultima frontiera della tecnologia cellulare.
Il suo principale scopo è consentire lo \gls{iot} massivo, ossia un \textit{network} che sia 
in grado di gestire la connessione di molti dispositivi con latenze molto piccole.
Per consentire velocità fino a 10 Gb/s si sono
dovute apportare importanti modifiche strutturali che rendono la sua architettura molto diversa da quelle viste fin'ora.\\
L'architettura implementata prende il nome di \gls{sba}.
La \gls{sba} consiste nel dividere tutti i componenti architetturali in una serie di \textit{microservices}\cite{5g-approach}. 
Questa nuova struttura è stata introdotta per garantire la scalabilità del sistema, migliorare le prestazioni (velocità) e per 
permettere la gestione simultanea di molti dispositivi.\\
I principali elementi che la compongono sono:
\begin{itemize}
    \item \gls{amf} responsabile dell'autenticazione e localizzazione del dispositivo.
    \item \gls{smf} per la gestione della sessione di ogni \gls{ms}.
    \item \gls{pcf} per la gestione delle \textit{policy}.
    \item \gls{udm} per la gestione dell'identità dell'utente, questo compito era precedentemente svolto da \gls{hss} o \gls{hlr}.
    \item \gls{ausf} per effettuare l'autenticazione dell'utente.
    \item \gls{sdsf} è un helper per la memorizzazione di dati strutturati.
    \item \gls{udsf} è un helper per la memorizzazione di dati non strutturati.
    \item \gls{nef} per esporre determinate funzionalità a servizi di terze parti.
    \item \gls{nrf} per scoprire tutti i servizi disponibili.
    \item \gls{nssf} per selezionare una determinata partizione di \textit{network}.
    \item \gls{upf} trasporta il traffico dal \gls{ran} all'internet.
\end{itemize}
\begin{figure}[ht]
    \centering
    \includegraphics[width=0.8\textwidth]{images/5g-planes.png}
    \caption{Architettura 5G\cite{5g-approach}}
\end{figure}

\clearpage

\subsection{Network Slicing}
Il \textit{Network Slicing} rappresenta una delle caratteristiche più importanti del 5G. Con questo termine si intende il partizionamento della
rete in diversi "piani" ciascuno con caratteristiche e requisiti particolari, indipendente e autonomo. Questo risulta fondamentale nella realizzazione 
dell' \gls{iot} massivo, infatti in questo modo la gestione del traffico terrà conto dell'applicazione che viene utilizzata nel dispositivo per decidere di quali prestazioni di rete ha bisogno. 
\begin{figure}[ht]
    \centering
    \includegraphics[width=0.8\textwidth]{images/5g-eg-of-use.png}
    \caption{Esempi di applicazioni per il 5G}
\end{figure}\\
Ogni segmento virtuale del \textit{network} ha uno specifico identificativo che deve essere indicato nella fase di autenticazione come verrà illustrato nella sezione 5.3. Per ogni \textit{slice} sono 
richieste delle prestazioni differenti, per esempio il settore delle \textit{critical communication} deve avere delle latenze molto basse.
\begin{figure}[ht]
    \centering
    \includegraphics[width=0.8\textwidth]{images/5g-slicing.jpg}
    \caption{\textit{Network slicing} nel 5G}
\end{figure}\\
La realizzazione del \textit{Network Slicing} avviene tramite il paradigma del \textit{Software Defined Network} che nella prossima sezione verrà approfondito.

\clearpage

\subsection{\textit{Software Defined Network} e  \textit{Network Functions Virtualization}}
Il \gls{sdn} è un paradigma per gestire il \textit{network} in modo efficace. 
In questo caso il \textit{network} è definito da \gls{nfv}, dove intere classi di funzioni sono virtualizzate.
Questi sono necessari per interfacciarsi a livello applicativo con i dispositivi cellulari 
in modo da gestire il traffico della rete in modo efficace\cite{5g-sdn}.