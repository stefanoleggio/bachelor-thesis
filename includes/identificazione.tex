\section{Sistema di identificazione}
Il meccanismo di identificazione è la procedura per verificare che un determinato dispositivo
è abilitato a connettersi alla rete.
Questo procedimento avviene tramite l'\textit{Authentication and key agreement} (AKA), procedimento in cui
il \textit{core network} abilita un dispositivo a connettersi. 


\subsection{2G}
\subsection{3G}
Con rete 3G si intendono l'insieme delle tecnologie di terza generazione, stiamo quindi parlando di un'architettura UMTS.
Un \acrshort{ms} che si vuole collegare alla rete deve procedere con la fase di autenticazione o identificazione anche detta \textit{Authentication and key agreement}
(AKA). In questa fase, viene interrogata la rispettiva HLR/AuC dove l'IMSI del dispositivo viene validato, se tutto procede correttamente
viene notificato il SGSN che inoltra al \acrshort{ms} l'avviso di autenticazione completata.
%Schema autenticazione umts

\subsection{4G}

\subsection{5G}