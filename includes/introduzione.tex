\section{Introduzione}
Le reti cellulari rappresentano un punto nevralgico per le nostre comunicazioni.
Per questo, la loro sicurezza è fondamentale per garantire un normale funzionamento
di tutti i servizi a cui ormai ci siamo abituati.\\
La nuova tecnologia di quinta generazione è ormai vicina ad essere implementata su larga scala
per permettere lo sviuluppo del mondo IOT \textit{Internet Of Things}. Questa nuova tecnologia stravolge numerosi
paradigmi strutturali che sono stati utilizzati fin'ora nelle generazioni precedenti, introducendo nuove sfide nell'ambito
della loro sicurezza.
\subsection{Struttura del documento}
Il documento è strutturato in modo da fornire al lettore le competenze e terminologie adeguate per comprendere tutti i dettagli della 
vulnerabilità scoperta.\\
L'elaborato inizia con una breve panoramica sulla rete cellulare, descrivendo genericamente la sua struttura e architettura.\\ 
Dato che le specifiche dell'architettura di una rete cellulare sono molto diverse a seconda della generazione, è stato 
necessario illustrare l'evoluzione delle varie tecnologie: da 1G a 5G. 
Per ogni generazione verranno illustratate prevalentemente le sue propietà architetturali oltre che le principali novità introdotte.
Successivamente, verrà introdotta la tipologia dell'attacco trattato, ossia il \textit{Denial of Service}, spiegano in cosa consiste
e come si applica alle reti cellulari. Inoltre, verranno illustrate le misurazioni necessarie per valutare l'efficienza di un attacco.\\
Nel seguente capitolo, verranno analizzati nel dettaglio i sistemi di identificazione per le varie generazioni cellulari. Questo perchè è nel 
loro funzionamento che sono preseti le vulnerabilità sfruttate per l'attacco.\\
Successivamente, verrà trattato l'attacco di tipo \textit{Denial of Service} alle reti UMTS, spiegando il suo funzionamento e i risultati che sono stati
ottenuti in \cite{umts-dos}.
Infine, verrà discusso una potenziale replicazione in una architettura 5g. Inoltre, verranno evidenziate altre possibili vulnerabilità presenti in questa 
ultima generazione.
\subsection{Scopo della tesi}
Questo elaborato si vuole occupare di analizzare l'attacco di tipo \textit{Denial of Service}
alle reti UMTS illustrato in \cite{umts-dos} e scoprire se questo potrebbe risultare efficacie nelle ultime
tecnologie cellulari 5g.